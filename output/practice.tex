% Options for packages loaded elsewhere
\PassOptionsToPackage{unicode}{hyperref}
\PassOptionsToPackage{hyphens}{url}
\PassOptionsToPackage{dvipsnames,svgnames,x11names}{xcolor}
%
\documentclass[
  11pt,
]{article}
\author{}
\date{\vspace{-2.5em}}

\usepackage{amsmath,amssymb}
\usepackage{lmodern}
\usepackage{iftex}
\ifPDFTeX
  \usepackage[T1]{fontenc}
  \usepackage[utf8]{inputenc}
  \usepackage{textcomp} % provide euro and other symbols
\else % if luatex or xetex
  \usepackage{unicode-math}
  \defaultfontfeatures{Scale=MatchLowercase}
  \defaultfontfeatures[\rmfamily]{Ligatures=TeX,Scale=1}
  \setmainfont[]{CMU Serif}
\fi
% Use upquote if available, for straight quotes in verbatim environments
\IfFileExists{upquote.sty}{\usepackage{upquote}}{}
\IfFileExists{microtype.sty}{% use microtype if available
  \usepackage[]{microtype}
  \UseMicrotypeSet[protrusion]{basicmath} % disable protrusion for tt fonts
}{}
\usepackage{xcolor}
\IfFileExists{xurl.sty}{\usepackage{xurl}}{} % add URL line breaks if available
\IfFileExists{bookmark.sty}{\usepackage{bookmark}}{\usepackage{hyperref}}
\hypersetup{
  colorlinks=true,
  linkcolor={blue},
  filecolor={Maroon},
  citecolor={Blue},
  urlcolor={Blue},
  pdfcreator={LaTeX via pandoc}}
\urlstyle{same} % disable monospaced font for URLs
\usepackage[margin=1in]{geometry}
\usepackage{graphicx}
\makeatletter
\def\maxwidth{\ifdim\Gin@nat@width>\linewidth\linewidth\else\Gin@nat@width\fi}
\def\maxheight{\ifdim\Gin@nat@height>\textheight\textheight\else\Gin@nat@height\fi}
\makeatother
% Scale images if necessary, so that they will not overflow the page
% margins by default, and it is still possible to overwrite the defaults
% using explicit options in \includegraphics[width, height, ...]{}
\setkeys{Gin}{width=\maxwidth,height=\maxheight,keepaspectratio}
% Set default figure placement to htbp
\makeatletter
\def\fps@figure{htbp}
\makeatother
\setlength{\emergencystretch}{3em} % prevent overfull lines
\providecommand{\tightlist}{%
  \setlength{\itemsep}{0pt}\setlength{\parskip}{0pt}}
\setcounter{secnumdepth}{5}
\usepackage{indentfirst}
\usepackage{fancyhdr}
\usepackage{cancel}
\pagestyle{fancy}
\fancyhead{}
\fancyhead[L]{\leftmark}
\newcommand{\R}{\mathbb{R}}
\newcommand{\Ll}{\mathbb{L}}
\newcommand{\N}{\mathbb{N}}
\newcommand{\F}{\mathbb{F}}
\setlength{\headheight}{15pt}
\ifLuaTeX
  \usepackage{selnolig}  % disable illegal ligatures
\fi

\begin{document}

\begin{titlepage}
\begin{center}
\bfseries

{\Large ФЕДЕРАЛЬНОЕ ГОСУДАРСТВЕННОЕ БЮДЖЕТНОЕ ОБРАЗОВАТЕЛЬНОЕ\\ УЧРЕЖДЕНИЕ ВЫСШЕГО ОБРАЗОВАНИЯ\\ "МОСКОВСКИЙ АВИАЦИОННЫЙ ИНСТИТУТ\\ (НАЦИОНАЛЬНЫЙ ИССЛЕДОВАТЕЛЬСКИЙ УНИВЕРСИТЕТ)"

}

\vspace{56pt}

{\large ЖУРНАЛ ПРАКТИКИ 

}

\end{center}

\vspace{36pt}

Студента 3 курса \hspace{3cm}        \underline{Гординского Дмитрия Михайловича}

\vspace{26pt}

Институт №8 \underline{«Компьютерные науки и прикладная математика»}

\vspace{26pt}

Кафедра №804 \underline{«Теория вероятностей и компьютерное моделирование»}

\vspace{26pt}

Учебная группа \underline{М8О-304Б-20}

\vspace{26pt}

Направление \underline{01.03.04} \hspace{3cm} \underline{Прикладная математика}

\vspace{26pt}

Вид практики \underline{Научно-исследовательская работа (получение первичных навыков научно \\
-исследовательской работы)}

\vspace{26pt}

Оценка за практику \underline{ \hspace{3cm}} Руководитель практики от МАИ \underline{Зайцева О.Б.}

\vspace{46pt}

\underline{Гординский Д.М / \hspace{3cm} / 7 июня 2023 г.}

\vfill

\begin{center}
\bfseries
Москва, \the\year
\end{center}
\end{titlepage}

\pagebreak


\textbf{1. Место и сроки проведения практики}

\vspace{20pt}

{\em Дата начала практики \hspace{3cm} 9 \underline{февраля} 2023 г.} \\

{\em Дата окончания практики \hspace{3cm} 7 \underline{июня} 2023 г.}

\vspace{20pt}

{\em Наименование организации \underline{МОСКОВСКИЙ АВИАЦИОННЫЙ ИНСТИТУТ(НИУ)}} \\

{\em Название структурного подразделения \underline{Кафедра 804}}

\vspace{20pt}

\textbf{2. Инструктаж по технике безопасности}

\vspace{10pt}

\underline{ Платонов Е. Н. / \hspace{3cm} /} \hspace{1cm} 9 \underline{февраля} 2023 г.

\vspace{10pt}

\textbf{3. Индивидуальное задание студенту}

Выбрать тему дипломной работы, ориентируясь на доступные наборы данных.

\vspace{10pt}

\textbf{4. План выполнения индивидуального задания}

\begin{itemize}
    \item[1.] Изучить теорию по Анализу выживаемости.
    \item[2.] Ознакомиться с необходимыми библиотеками для работы с данными и их графическим представлением.
    \item[3.] Ознакомиться с датасетами.
    \item[4.] Выбрать тему работы и датасет на основании изученной информации.
    \item[5.] Написать реферат.
\end{itemize}

\textbf{Утверждаю}

{\em Руководитель практики от МАИ: \underline{Зайцева О.Б. / \hspace{3cm} /}} 9 \underline{февраля} 2023 г.

\vspace{20pt}

\underline{Платонов Е. Н. / \hspace{3cm} /} 9 \underline{февраля} 2023 г.

\vspace{10pt}

\textbf{Ознакомлен}

\underline{Гординский Д. М. / \hspace{3cm} /} 9 \underline{февраля} 2023 г.

\pagebreak

\textbf{5. Отзыв руководителя практики}

{\em Задание на практику выполнено в полном объеме. Материалы, изложенные в отчете студента, полностью соответствуют индвидуальному заданию. Рекомендую оценку отлично. }

\vspace{30pt}

{\em Руководитель \hspace{3cm} \underline{ Платонов Е. Н. / \hspace{3cm} /} 7 \underline{июня} 2023 г.}

\pagebreak


\tableofcontents

\pagebreak

\hypertarget{ux432ux432ux435ux434ux435ux43dux438ux435}{%
\section{Введение}\label{ux432ux432ux435ux434ux435ux43dux438ux435}}

\textbf{\emph{Анализ выживаемости}} --- это статистический метод,
используемый для изучения времени, которое прошло до наступления
определенного события. Этот метод может использоваться для анализа
многих видов данных, включая медицинские, социальные и экономические.

В последнее время, анализ выживаемости стал одним из наиболее популярных
методов исследования. Это связано с тем, что данный метод позволяет
проводить анализы с использованием цензурированных данных. В свою
очередь, это позволяет учитывать не только количество произошедших
событий, но и время, прошедшее до наступления события.

Одной из наиболее важных областей, в которых используется анализ
выживаемости, является медицина, где он используется для анализа
выживаемости пациентов после лечения различных заболеваний. Кроме того,
анализ выживаемости используется в экономике, социологии, и многих
других областях.

В этой работе я хотел бы рассмотреть конкретно сферу медицины.

Выбор данных для анализа --- это важный шаг в любом исследовании.

Из-за того, что работа с пациентом подразумевает полный набор данных и
характеристик, то в сфере медицинских исследований мы чаще всего можем
получить большую репрезентативную выборку с множеством параметров.

Имеено по этой причине я выбрал датасет клиники Майо по первичному
билиарному циррозу

\hypertarget{ux43aux440ux430ux442ux43aux43eux435-ux43eux43fux438ux441ux430ux43dux438ux435-ux434ux430ux43dux43dux44bux445}{%
\section{Краткое описание
данных}\label{ux43aux440ux430ux442ux43aux43eux435-ux43eux43fux438ux441ux430ux43dux438ux435-ux434ux430ux43dux43dux44bux445}}

Эти данные взяты из клинического исследования первичного билиарного
цирроза (ПБЦ) печени, проводившегося в клинике Майо в период с 1974 по
1984 год. В общей сложности 424 пациента с ПБЦ, направленные в клинику
Майо в течение этого десятилетнего интервала, соответствовали критериям
отбора для рандомизированного исследования плацебо. контролируемое
исследование препарата D-пеницилламин. Первые 312 случаев в наборе
данных участвовали в рандомизированном исследовании и содержат в
основном полные данные. Дополнительные 112 пациентов не участвовали в
клинических испытаниях, но дали согласие на запись основных измерений и
наблюдение за их выживаемостью. Шесть из этих случаев были потеряны для
последующего наблюдения вскоре после постановки диагноза, поэтому данные
здесь относятся к дополнительным 106 случаям, а также к 312
рандомизированным участникам.

\hypertarget{ux43eux43fux438ux441ux430ux43dux438ux435-ux43fux43eux43bux435ux439}{%
\section{Описание
полей}\label{ux43eux43fux438ux441ux430ux43dux438ux435-ux43fux43eux43bux435ux439}}

\begin{verbatim}
age:    in years
albumin:    serum albumin (g/dl)
alk.phos:   alkaline phosphotase (U/liter)
ascites:    presence of ascites
ast:    aspartate aminotransferase, once called SGOT (U/ml)
bili:   serum bilirunbin (mg/dl)
chol:   serum cholesterol (mg/dl)
copper: urine copper (ug/day)
edema:  0 no edema, 0.5 untreated or successfully treated
1 edema despite diuretic therapy
hepato: presence of hepatomegaly or enlarged liver
id: case number
platelet:   platelet count
protime:    standardised blood clotting time
sex:    m/f
spiders:    blood vessel malformations in the skin
stage:  histologic stage of disease (needs biopsy)
status: status at endpoint, 0/1/2 for censored, transplant, dead
time:   number of days between registration and the earlier of death,
transplantion, or study analysis in July, 1986
trt:    1/2/NA for D-penicillmain, placebo, not randomised
trig:   triglycerides (mg/dl)
\end{verbatim}

\hypertarget{ux43e-ux43fux440ux435ux434ux43cux435ux442ux435-ux438ux441ux441ux43bux435ux434ux43eux432ux430ux43dux438ux44f}{%
\subsection{О предмете
исследования}\label{ux43e-ux43fux440ux435ux434ux43cux435ux442ux435-ux438ux441ux441ux43bux435ux434ux43eux432ux430ux43dux438ux44f}}

\textbf{\emph{Цирроз печени}} ---- это заболевание, характеризующееся
перерождением паренхиматозной ткани печени в фиброзную соединительную
ткань. Сопровождается тупой болью в правом подреберье, желтухой,
повышением давления в системе воротной вены с характерными для
портальной гипертензии кровотечениями (пищеводными, геморроидальными),
асцитом и пр. Заболевание носит хронический характер. В диагностике
цирроза печени определяющую роль играют данные УЗИ, КИ и МРТ печени,
показатели биохимических проб, биопсия печени. Лечение цирроза печени
предусматривает строгий отказ от алкоголя, соблюдение диеты, прием
гепатопротекторов; в тяжелых случаях ---- трансплантацию донорской
печени.

\hypertarget{ux43fux440ux438ux447ux438ux43dux44b-ux432ux43eux437ux43dux438ux43aux43dux43eux432ux435ux43dux438ux44f}{%
\subsubsection{Причины
возникновения}\label{ux43fux440ux438ux447ux438ux43dux44b-ux432ux43eux437ux43dux438ux43aux43dux43eux432ux435ux43dux438ux44f}}

Цирроз характеризуется возникновением в ткани печени
соединительнотканных узлов, разрастанием соединительной ткани,
формированием «ложных» долек. Цирроз различают по размеру формирующихся
узлов на мелкоузловой (множество узелков до 3 мм в диаметре) и
крупноузловой (узлы превышают 3 мм в диаметре). Изменения структуры
органа в отличие от гепатитов необратимы, таким образом, цирроз печени
относится к неизлечимым заболеваниям.

Среди причин развития цирроза печени лидирует злоупотребление алкоголем
(от 35,5\% до 40,9\% пациентов). На втором месте располагается вирусный
гепатит С. У мужчин цирроз развивается чаще, чем у женщин, что связано с
большим распространением в мужской среде злоупотребления алкоголем.

В подавляющем большинстве случаев причиной развития цирроза печени
является злоупотребление алкоголем и вирусные гепатиты В и С, реже -
ферментопатии:

\begin{itemize}
\item
  Алкогольная зависимость. Регулярное употребление алкоголя в дозах
  80-160 мл этанола ведет к развитию алкогольной болезни печени, которая
  в свою очередь прогрессирует с возникновением цирроза. Среди лиц,
  злоупотребляющих алкоголем на протяжении 5-10 лет, циррозом страдает
  35\%.
\item
  Заболевания гепатобилиарной системы. Хронические гепатиты также
  зачастую ведут к фиброзному перерождению ткани печени. На первом месте
  по частоте диагностирования стоят вирусные гепатиты В и С (гепатит С
  склонен к более деструктивному течению и прогрессирует в цирроз чаще).
  Также цирроз может стать результатом хронического аутоиммунного
  гепатита, склерозирующего холангита, первичного холестатического
  гепатита, сужения желчных протоков, застоя желчи. Циррозы,
  развивающиеся вследствие нарушений в циркуляции желчи, называют
  билиарными. Они подразделяются на первичные и вторичные.
\item
  Метаболические нарушения. Причиной развития цирроза печени может стать
  обменная патология или недостаточность ферментов: муковисцидоз,
  галактоземия, гликогеноз, гемохроматоз.
\end{itemize}

\hypertarget{ux444ux430ux43aux442ux43eux440ux44b-ux440ux438ux441ux43aux430}{%
\subsubsection{Факторы
риска}\label{ux444ux430ux43aux442ux43eux440ux44b-ux440ux438ux441ux43aux430}}

К факторам риска перерождения печеночной ткани относят:

\begin{itemize}
\tightlist
\item
  гепатолентикулярную дегенерацию (болезнь Вильсона);
\item
  прием гепатотоксичных лекарственных препаратов (метотрексат,
  изониазид, амиодарон, метил-допа);
\item
  хроническую сердечную недостаточность;
\item
  синдром Бада-Киари;
\item
  операционные вмешательства на кишечнике;
\item
  паразитарные поражения кишечника и печени.
\item
  В 20-30\% случаев причину развития цирроза печени установить не
  удается, такие циррозы называют криптогенными.
\end{itemize}

Таким образом, мы понимаем, что данная болезнь может подразумевать
глубое исследование.

\hypertarget{ux43fux43bux430ux43d-ux440ux430ux431ux43eux442-ux438-ux432ux44bux432ux43eux434ux44b}{%
\section{План работ и
выводы}\label{ux43fux43bux430ux43d-ux440ux430ux431ux43eux442-ux438-ux432ux44bux432ux43eux434ux44b}}

Из-за большого количества параметров в датасете мы можем выбрать разные
методы исследования проблемы, в нашем случае --- болезни.

Поэтому на основании данных о пациентах, которые болели циррозом печени,
может быть Тема \textbf{\emph{``Какие факторы оказывают наибольшее
влияние на выживаемость пациентов с ПБЦ печени''}}.

В рамках этой темы можно провести анализ связи между различными
факторами, такими как пол, возраст, наличие других заболеваний, уровень
печеночных ферментов и т.д., и шансами выживания пациентов с ПБЦ. Это
поможет понять, какие факторы могут быть наиболее важными при принятии
решений о лечении этой болезни, и какие методы лучше использовать для
запуска программ лечения.

\hypertarget{ux43cux435ux442ux43eux434ux438ux43aux430-ux440ux430ux431ux43eux442ux44b}{%
\subsection{Методика
работы}\label{ux43cux435ux442ux43eux434ux438ux43aux430-ux440ux430ux431ux43eux442ux44b}}

Для анализа датасета, я бы использовал язык программирования Python в
среде разработки Jupyter Notebook и некоторые библиотеки для работы с
данными, такие как pandas, matplotlib и seaborn.

Либо я бы использовал язык R из-за возможности более просто выводить
графики

\hypertarget{ux432ux44bux432ux43eux434}{%
\subsubsection{Вывод}\label{ux432ux44bux432ux43eux434}}

\textbf{\emph{Чем данная данная работа может быть полезна?}}

После выявляения факторов, можно использовать эти данные для повышения
качества медицинской помощи и объяснения пациентам потенциального исхода
лечения. Возможно, в дальнейшем, будут разработаны более точные
алгоритмы предсказания выживаемости пациентов, больных циррозом печени.

\hypertarget{ux438ux441ux442ux43eux447ux43dux438ux43aux438}{%
\section{Источники}\label{ux438ux441ux442ux43eux447ux43dux438ux43aux438}}

\begin{itemize}
\item
  \url{https://scikit-survival.readthedocs.io/en/stable/user_guide/00-introduction.html}
\item
  \url{https://lifelines.readthedocs.io/en/latest/index.html}
\item
  \url{https://paperswithcode.com/task/survival-analysis}
\item
  \url{https://www-eio.upc.edu/~pau/cms/rdata/doc/survival/pbc.html}
\item
  \url{http://statsoft.ru/home/textbook/modules/stsurvan.html\#general}
\end{itemize}

\end{document}
